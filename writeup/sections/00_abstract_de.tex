\selectlanguage{ngerman}
\chapter*{Kurzfassung}
\markboth{Abstract}{Abstract}
Die Optimierung der Strukturtopologie ist mittlerweile ein gut entwickeltes Werkzeug, mit dem viele Ingenieur:innen problemlos optimale Strukturen für minimales Gewicht ermitteln können. Die Optimierung der thermischen Topologie wird ebenfalls zu einem gut untersuchten Thema, und auch werden viele Methoden derzeit entwickelt. Durch die Integration dieser Tools in ein einziges Modul kann theoretisch eine Struktur gefunden werden, die beide Probleme optimiert. Normalerweise führen diese zu einem gewissen Kompromiss zwischen den einzelnen Problemen, die berücksichtigt werden müssen. Die Idee der Pareto-Optimalität entsteht, wenn eine Struktur gefunden werden kann, die den Fehler der Struktur in Bezug auf einige nicht erreichbare Entwurfskriterien minimiert.

Natürlich müssen die Materialeigenschaften bekannt sein, um den Bereich zu optimieren. Für viele Gitterelementarzellen wurden analytische und semianalytische Modelle entwickelt. Diese Modelle werden in dieser Arbeit verwendet.

In dieser Arbeit wurde eine Methodik zur multifunktionalen Topologieoptimierung entwickelt sowie eine grundlegende Implementierung in der Programmiersprache "Julia" geschrieben. Die Finite-Elemente-Methode wurde zur Lösung struktureller, thermischer und Phasenänderungsprobleme eingesetzt. Eine optimale Struktur kann anhand eines sogenannten Utopiepunkts ermittelt werden, der in der Regel nicht exakt gefunden werden kann. Es wurden auch Werkzeuge zur Lösung einfunktionaler Topologieoptimierungsprobleme implementiert, die die homogenisierten Eigenschaften ausgewählter Gitterelementarzellen berücksichtigen.

\textbf{Schlagwörter:} Gitterstrukturen, Phasenwechselmaterialien, Multifunktionalität, Topologieoptimierung, Pareto-Optimalität