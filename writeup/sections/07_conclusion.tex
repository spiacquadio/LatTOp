\chapter{Conclusion}
\label{chap:conclusion}
\section{Conclusion}
In this thesis, a methodology to produce multi-functional topology-optimized structures has been developed and investigated. Structural, thermal and phase change considerations have been incorporated, and the results they produce have been discussed. While this thesis focussed on their use in spacecraft structures, the methodology can also be applied to Earth-bound systems.

The thesis began by looking into the theoretical background of all the ingredients required. Heat transfer and the Stefan problem were the roots of what would be built on. Structural mechanics was also the most important topic that was required to produce traditional topology-optimized structures. Finally, optimization methods were discussed which would be used throughout this thesis. While automatic differentiation was not as useful as hoped, it was implemented in many stages for computing certain gradients efficiently.

The state of the art was then researched which would provide the groundwork for what this thesis would build on. Homogenized properties for the cellular structures were provided that were used in all topology optimizations. Latent heat energy storage materials were found, and paraffin was selected for this thesis as it was a simple material to implement that could be widely varied for specific use cases. Topology optimization algorithms were found, most importantly the structural and thermal algorithms. Multifunctional topology optimization methods were found in the literature that could be adapted to this specific case.

Finally, a methodology for the structural-thermal and structural-phase-change optimization has been developed. The algorithms are so general that they can be applied to any material or any homogenized unit cell. Automatic differentiation, while not useful for solving the full script, was extremely useful and efficient in computing element matrices and simple program derivatives. Some test cases were also investigated to ensure different boundary conditions could be solved.

\section{Discussion}
While This thesis developed and investigated a methodology for multi-functional optimization, it did not explicitly consider the de-homogenization of the produced structure. The structures that are produced contain elements much smaller than a unit cell. This means that their reactive density needs to be somehow averaged over the volume of a unit cell in a de-homogenization process. This will naturally produce a different stiffness and maximum temperature.

This thesis was also focused on space-craft structures. This means the phase change simulation only considers conduction and no convection. Topology optimization has been successfully performed considering the Navier-Stokes equations \cite{Alexandersen_Sigmund_Aage_2016}, but this time cost is significant. In future work, phase change optimized structures should consider convection to be able to produce highly optimized structures for use on Earth. Such structures could be used for energy storage as a low-cost and efficient alternative to chemical batteries.